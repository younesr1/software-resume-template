%-------------------------------------
% LaTeX Resume for Software Engineers
% Author : Leslie Cheng
% License : MIT
%-------------------------------------

\documentclass[letterpaper,12pt]{article}[leftmargin=*]

\usepackage[empty]{fullpage}
\usepackage{enumitem}
\usepackage{ifxetex}
\ifxetex
  \usepackage{fontspec}
  \usepackage[xetex]{hyperref}
\else
  \usepackage[utf8]{inputenc}
  \usepackage[T1]{fontenc}
  \usepackage[pdftex]{hyperref}
\fi
\usepackage{fontawesome}
\usepackage[sfdefault,light]{FiraSans}
\usepackage{anyfontsize}
\usepackage{xcolor}
\usepackage{tabularx}

%-------------------------------------------------- SETTINGS HERE --------------------------------------------------
% Header settings
\def \fullname {Younes Reda}
\def \subtitle {2B Mechatronics Engineering}

\def \linkedinicon {\faLinkedin}
\def \linkedinlink {https://linkedin.com/in/younes-reda123/}
\def \linkedintext {/younes-reda123}

\def \phoneicon {\faPhone}
\def \phonetext {647-968-9360}

\def \emailicon {\faEnvelope}
\def \emaillink {mailto:yreda@waterloo.ca}
\def \emailtext {yreda@uwaterloo.ca}

\def \githubicon {\faGithub}
\def \githublink {https://github.com/younesr1}
\def \githubtext {/younesr1}

\def \headertype {\doublecol} % \singlecol or \doublecol

% Misc settings
\def \entryspacing {-0pt}

\def \bulletstyle {\faAngleRight}

% Define colours
\definecolor{primary}{HTML}{000000}
\definecolor{secondary}{HTML}{0D47A1}
\definecolor{accent}{HTML}{263238}
\definecolor{links}{HTML}{1565C0}

%------------------------------------------------------------------------------------------------------------------- 

% Defines to make listing easier
\def \linkedin {\linkedinicon \hspace{3pt}\href{\linkedinlink}{\linkedintext}}
\def \phone {\phoneicon \hspace{3pt}{ \phonetext}}
\def \email {\emailicon \hspace{3pt}\href{\emaillink}{\emailtext}}
\def \github {\githubicon \hspace{3pt}\href{\githublink}{\githubtext}}
\def \website {\websiteicon \hspace{3pt}\href{\websitelink}{\websitetext}}

% Adjust margins
\addtolength{\oddsidemargin}{-0.55in}
\addtolength{\evensidemargin}{-0.55in}
\addtolength{\textwidth}{1.1in}
\addtolength{\topmargin}{-0.6in}
\addtolength{\textheight}{1.1in}

% Define the link colours
\hypersetup{
    colorlinks=true,
    urlcolor=links,
}

% Set the margin alignment 
\raggedbottom
\raggedright
\setlength{\tabcolsep}{0in}

%-------------------------
% Custom commands

% Sections
\renewcommand{\section}[2]{\vspace{5pt}
  \colorbox{secondary}{\color{white}\raggedbottom\normalsize\textbf{{#1}{\hspace{7pt}#2}}}
}

% Entry start and end, for spacing
\newcommand{\resumeEntryStart}{\begin{itemize}[leftmargin=2.5mm]}
\newcommand{\resumeEntryEnd}{\end{itemize}\vspace{\entryspacing}}

% Itemized list for the bullet points under an entry, if necessary
\newcommand{\resumeItemListStart}{\begin{itemize}[leftmargin=4.5mm]}
\newcommand{\resumeItemListEnd}{\end{itemize}}

% Resume item
\renewcommand{\labelitemii}{\bulletstyle}
\newcommand{\resumeItem}[1]{
  \item\small{
    {#1 \vspace{-2pt}}
  }
}

% Entry with title, subheading, date(s), and location
\newcommand{\resumeEntryTSDL}[4]{
  \vspace{-1pt}\item[]
    \begin{tabularx}{0.97\textwidth}{X@{\hspace{60pt}}r}
      \textbf{\color{primary}#1} & {\firabook\color{accent}\small#2} \\
      \textit{\color{accent}\small#3} & \textit{\color{accent}\small#4} \\
    \end{tabularx}\vspace{-6pt}
}

% Entry with title and date(s)
\newcommand{\resumeEntryTD}[2]{
  \vspace{-1pt}\item[]
    \begin{tabularx}{0.97\textwidth}{X@{\hspace{60pt}}r}
      \textbf{\color{primary}#1} & {\firabook\color{accent}\small#2} \\
    \end{tabularx}\vspace{-6pt}
}

% Entry for special (skills)
\newcommand{\resumeEntryS}[2]{
  \item[]\small{
    \textbf{\color{primary}#1 }{ #2 \vspace{-6pt}}
  }
}

% Double column header
\newcommand{\doublecol}[6]{
  \begin{tabularx}{\textwidth}{Xr}
    {
      \begin{tabular}[c]{l}
        \fontsize{35}{45}\selectfont{\color{primary}{{\textbf{\fullname}}}} \\
        {\textit{\subtitle}} % You could add a subtitle here
      \end{tabular}
    } & {
      \begin{tabular}[c]{l@{\hspace{1.5em}}l}
        {\small#4} & {\small#1} \\
        {\small#5} & {\small#2} \\
        {\small#6} & {\small#3}
      \end{tabular}
    }
  \end{tabularx}
}

% Single column header
\newcommand{\singlecol}[6]{
  \begin{tabularx}{\textwidth}{Xr}
    {
      \begin{tabular}[b]{l}
        \fontsize{35}{45}\selectfont{\color{primary}{{\textbf{\fullname}}}} \\
        {\textit{\subtitle}} % You could add a subtitle here
      \end{tabular}
    } & {
      \begin{tabular}[c]{l}
        {\small#1} \\
        {\small#2} \\
        {\small#3} \\
        {\small#4} \\
        {\small#5} \\
        {\small#6}
      \end{tabular}
    }
  \end{tabularx}
}

\begin{document}
%-------------------------------------------------- BEGIN HERE --------------------------------------------------

%---------------------------------------------------- HEADER ----------------------------------------------------

\headertype{\linkedin}{\github}{\website}{\phone}{\email}{} % Set the order of items here
\vspace{0pt} % Set a negative value to push the body up, and the opposite

%-------------------------------------------------- EXPERIENCE --------------------------------------------------
\section{\faPieChart}{Experience}

  \resumeEntryStart
    \resumeEntryTSDL
      {Siemens Health}{May 2020 -- Sept 2020}
      {Embedded Software Developer}{Ottawa, ON}
    \resumeItemListStart
      \resumeItem {Prototyped USB OTG functionality for STM32H7 dual core processor, opening markets with stringent wireless rules (e.g military hospitals)}
      \resumeItem {Designed and implemented self-correcting redundancy of product critical data structures, reducing likelihood of product failure}
      \resumeItem {Configured processors through CubeMx and debugged with IAR and logic analyzers}
      \resumeItem {Ported I2C, SPI, QSPI, and sensor specific drivers from F7 to H7 for the new generation product}
      \resumeItem {Worked directly with HAL libraries, taking advantage of the code generation feature of CubeMx to produce reliable code}
    \resumeItemListEnd
  \resumeEntryEnd

  \resumeEntryStart
    \resumeEntryTSDL
      {Waterloo Robotics Team}{Jan 2020 -- Present}
      {Firmware Team Lead: \href{https://github.com/uwrobotics/MarsRover2020-firmware}{github}}{Waterloo, ON}
    \resumeItemListStart
      \resumeItem {Guiding all firmware development for on-board sensors and PCBs and leading a team of 10}
      \resumeItem {Enforcing modern C++20 coding standards and adhering to the \emph{C++ Core Guidelines}}
      \resumeItem {Configured 3-way load balancing on 5GHz, 2.4GHz, and 900MHz bands to ensure a highly-adaptable communications system}
      \resumeItem {Designed cascaded PID control of robotic arm with feed-forward term, improving arm dexterity}
      \resumeItem {Implemented PID tuning of rover's arm actuators via CAN for rapid and convenient tuning}
    \resumeItemListEnd
  \resumeEntryEnd

  \resumeEntryStart
    \resumeEntryTSDL
      {Ross Video}{Sept 2019 -- Aug 2019}
      {Test Automation Developer}{Ottawa, ON}
    \resumeItemListStart
        \resumeItem {Developed automation scripts for testing signal processing hardware running Embedded Linux, cutting test-time by half}
        \resumeItem {Kickstarted WebUI automation effort and automated 75\% of test suite}
        \resumeItem {Acquired a strong understanding of IP networking fundamentals by conducting cross network packet transferring}
    \resumeItemListEnd
  \resumeEntryEnd

%-------------------------------------------------- PROJECTS --------------------------------------------------
\section{\faFlask}{Projects}

  \resumeEntryStart
    \resumeEntryTD
      {Jetbot Localizer and Mapper : \href{https://github.com/younesr1/jetbot_slam}{github}}{}
    \resumeItemListStart
      \resumeItem {Built a wheeled robot based off NVIDIA's Jetbot platform to gain experience with openCV and modern C++}
      \resumeItem {Programmed remote operation  of the robot via Bluetooth with live camera recording}
      \resumeItem {Implemented cross frame feature tracking based on the K-nearest neighbours matching algorithm with noise filtering}
      \resumeItem {Currently investigating camera pose estimation by calculating intrinsic and extrinsic matrices of camera}
      \resumeItem {Continuing development with the ultimate goal being 3D point-cloud reconstruction of the robot's environment with pose estimation}
    \resumeItemListEnd
  \resumeEntryEnd

%-------------------------------------------------- Toolbox ----------------------------------------------------
\section{\faGears}{Toolbox}
 \resumeEntryStart
  \resumeEntryS{Languages } {C, C++, Python, MatLab, CMake, Bash}
  \resumeEntryS{Embedded Development} {Ubuntu, IAR, Keil µVision, CubeMx, Logic Analyzer, GDB, GCC}
  \resumeEntryS{Hardware} {STM32(H7, F7, F4), NVIDIA Jetson Nano, Raspberry Pi, Arduino, LPC1768}
  \resumeEntryS{Libraries } {OpenCV, MbedOS, STL, Boost}
  \resumeEntryS{Collaboration } {Git, Github, SVN, Jira, Azure}
 \resumeEntryEnd

%-------------------------------------------------- EDUCATION --------------------------------------------------
\section{\faGraduationCap}{Education}
  \resumeEntryStart
    \resumeEntryTSDL
      {University of Waterloo}{2018 -- 2023}
      {BSc. Mechatronics Engineering}{Waterloo, ON}
  \resumeEntryEnd

\end{document}
